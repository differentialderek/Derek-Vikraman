\documentclass{article}
\usepackage[utf8]{inputenc}
\usepackage{xcolor,amsfonts,amsmath,amsthm,amssymb,enumitem,hyperref,comment} 
\usepackage{tikz-cd}

\newcommand{\aut}{\textrm{Aut}}
\newcommand{\C}{$\mathcal{C} \:$}
\newcommand{\Cc}{$\mathcal{C}$}
\newcommand{\Cm}{\mathcal{C}}
\newcommand{\univ}{\mathcal{U}}
\newcommand{\question}[1]{\noindent \textcolor{red}{\textbf{Question:} #1}}
\newcommand{\note}{\noindent \textbf{\textit{Note:}} $\:$}
\newcommand{\define}[1]{\noindent \textbf{Definition.} \textit{(#1)} $\:$}
\newcommand{\n}{\mathbb{N}}
\newcommand{\z}{\mathbb{Z}}
\newcommand{\Univ}{\mathcal{U}}
\newcommand{\Univfg}{\mathcal{U}_{\textrm{fg}}}
\newcommand{\Univf}{\mathcal{U}_{\textrm{f}}}
\newcommand{\0}{\mathbf{0}}
\newcommand{\1}{\mathbf{1}}
\newcommand{\defequiv}{: \equiv}
\newcommand{\defeq}{:=}
\newcommand{\inv}{^{-1}}
\newcommand{\eone}{^1}
\newcommand{\etwo}{^2}
\newcommand{\pr}{^\prime}
\newcommand{\fg}{\z_}%FG_{ab}}
\newcommand{\fib}{\textrm{fib}}
\newcommand{\inj}{\hookrightarrow}
\newcommand{\surj}{\twoheadrightarrow}
\newcommand{\sone}{\mathbb{S}^1}
\newcommand{\stwo}{\mathbb{S}^2}
\newcommand{\sthree}{\mathbb{S}^3}
\newcommand{\base}{\textrm{base}}
\newcommand{\Loop}{\textrm{loop}}
\newcommand{\cone}{\textrm{cone}}
\newcommand{\SES}{\textrm{SE}\mathbb{S}}
\newcommand{\isexact}{\textrm{is-exact}}
\newcommand{\V}{\textrm{V}}
\newcommand{\ap}[1]{\textrm{ap}_{#1}}
\newcommand{\id}{\textrm{id}}
\newcommand{\TODO}[1]{\marginpar{\textcolor{blue}{#1}}}
\newcommand{\refl}{\textrm{refl}}

\newtheorem{lemma}{Lemma}


\title{Serre Classes in Homotopy Type Theory}
\date{February 2020}

\begin{document}

\maketitle

\section{Introduction}
Serre classes are important in ... 

\section{Free Group and Free Abelian Group}

Free groups and free abelian groups can be defined with the functions:

$$ \otimes_\_ \sone : \n \to \Univ, 0 \mapsto \1, (n+1) \mapsto \sone \times (\otimes_n \sone) $$


$$ \vee_\_ \sone : \n \to \Univ, 0 \mapsto \1, (n+1) \mapsto \sone \times (\vee_n \sone) $$

Then the type representing finitely generated abelian groups is 
$$ \Univfg \defequiv \Sigma_{B : \Univ} || \Sigma_{n : \n} \Sigma_{f : \otimes_n \sone \to B} \textrm{is-surj}(\ap{f}) ||_{-1} $$
and the type representing finitely generated groups is 
$$ \Univf \defequiv \Sigma_{B : \Univ} || \Sigma_{n : \n} \Sigma_{f : \vee_n \sone \to B} \textrm{is-surj}(\ap{f}) ||_{-1} $$

\section{Exactness and Exact Sequences}
A \textit{group} is defined to be a pointed, 1-connected type $(B,b_0)$ such that $(b_0 = b_0)$ is a set. A \textit{short exact sequence} of abelian \TODO{Does it make sense to take out ``abelian''?} groups is a sequence of maps $$ \1 \to A \to B \to C \to \1 $$ such that for the induced maps $$ (* = *) \to (a_0 = a_0) \to (b_0 = b_0) \to (c_0 = c_0) \to (* = *)$$ the kernel of each map is equivalent to the image of the map before it.

\section{$\Univfg$ is a Serre Class}
Consider the short exact sequence $$ \1 \to A \to B \to C \to \1.$$ We will show that there is a witness $\omega_B$ such that $(B, \omega_B) : \Univfg$ if and only if there are witnesses $\omega_A$ and $\omega_C$ such that $(A, \omega_A) : \Univfg$ and $(C, \omega_C) : \Univfg$. This shows that $\Univfg$ has the properties classically known to be those of \textit{Serre classes}.

To that end, let $(B, \omega_B) : \Univfg$. Then $\omega_B$ gives a map $\pi_B : \otimes_n \sone \to B$ such that $\ap{\pi_B}$ surjects. To show our result we need to provide maps $\pi_A$ and $\pi_C$ such that $\ap{\pi_A}$ and $\ap{\pi_C}$ are surjections. The latter of these we get for free by composing $$ \otimes_n \sone \to B \to C.$$
To construct $\pi_A$, consider the pullback square
\begin{center}
\begin{tikzcd}
    %A \times_B \otimes_n \sone 
    P \arrow[d,swap,"\ap{g}^*"] \arrow[r,"\ap{f}^*"] & \base =_{\otimes_n \sone} \base \arrow[d,"\ap{g}",twoheadrightarrow] \\
    a_0 =_A a_0 \arrow[r,swap,hook,"\ap{f}"] & b_0 =_B b_0
\end{tikzcd}
\end{center}
such that $\ap{f}$ is an embedding and $\ap{g}$ is a surjection. Applying Lemma \ref{lemma: pullback} to ap$_f$ and ap$_g$, we get $\textrm{inj}_A : P \inj (\base =_{\otimes_n \sone} \base)$ and $\textrm{surj}_A : P \surj (a_0 =_A a_0)$. Then \textcolor{red}{Lemma \ref{lemma: subgroup of free abelian}} gives us an inhabitant of $P \cong (\base =_{\otimes_k \sone} \base)$ for some $k : \n$. We construct our desired map $\omega_A\pr : \otimes_k \sone \to A$ via $\textrm{surj}_A$, yielding $\omega_A$.

Going the other way, given $(A, \omega_A)$ and $(C, \omega_C)$, we get the diagram

\begin{center}
\begin{tikzcd}
    (\base =_{\otimes_{k\prime} \sone} \base) \arrow[d,twoheadrightarrow] & & (\base =_{\otimes_{k^{\prime\prime\prime}} \sone} \base) \arrow[d,twoheadrightarrow] \\
    (a_0 =_A a_0) \arrow[r,hook] & (b_0 =_B b_0) \arrow[r,twoheadrightarrow] & (c_0 =_C c_0)
\end{tikzcd}
\end{center}
To construct a map in $\otimes_{k^{\prime\prime}} \sone \to B$, $k^{\prime\prime} = k\pr + k^{\prime \prime \prime}$, which yields a surjection in $(\base =_{\otimes_{k^{\prime\prime\prime}} \sone} \base) \to (b_0 =_B b_0)$, by \textcolor{red}{Lemma \ref{lemma: group extension}} we have $$(b_0 =_B b_0) = \Sigma_{c : (c_0 =_C c_0)} (a_0 =_A a_0).$$Then the surjection in $$(\base =_{\otimes_{k^{\prime\prime\prime}} \sone} \base) \to (b_0 =_B b_0)$$ is just given by composition with \textcolor{red}{\textbf{the} equivalence in $$(\base =_{(\otimes_{k^{\prime}} \sone \times \otimes_{k^{\prime\prime\prime}} \sone)} \base) \cong (\base =_{\otimes_{k^{\prime\prime\prime}} \sone} \base)$$}and $(s_1,s_2) \mapsto (\ap{\pi_A}(s_1), \ap{\pi_C}(s_2))$, where $\ap{\pi_A},\ap{\pi_C}$ are, respectively, the surjections into $(a_0 =_A a_0)$ and $(c_0 =_C c_0)$.

%\textcolor{red}{Note that this construction defines a mapping between particular surjections, so you may be able to get away without truncating (?).}

\begin{lemma}\label{lemma: pullback}
Pullbacks preserve surjections and embeddings.
\end{lemma}
\begin{proof}
We consider a pullback square 

\begin{center}
\begin{tikzcd}
    A \times_B C \arrow[r,"f^*"] \arrow[d,swap,"g^*"] & C \arrow[d, "g"] \\
    A \arrow[r,swap,"f"] & B
\end{tikzcd}
\end{center}
First consider $g$ and $g^*$, and suppose that $g$ is a surjection. We show that $g^*$ is a surjection. We use $ \textrm{is-surj} : (X \to Y) \to \textrm{hProp}$ defined by $\phi \mapsto \Pi_{y : Y} || \textrm{fib}_\phi(y) ||_{-1}$. To inhabit is-surj$(g^*)$, take $a : A$. An element of the fiber of $a$ over $g^*$ will have the form $(a,b,\omega_{(a,b)})$. Since $g$ surjects, there is some $b : B$ with $\omega: f(a) = g(b)$. Since we only need to show mere inhabitation, the choice of $b$ doesn't matter and we have our preimage.

Now consider $f^*$, and suppose that $f$ is an embedding. We want to show is-embed$(\phi) \defequiv (\Pi_{x_1, x_2} f^*(x_1) = f^*(x_2)) \to x_1 = x_2) \times \textrm{is-equiv(ap}_{f^*})$ with $x_i \defequiv (a_i, b_i, \omega_i)$. A path in $f^*(x_1) = f^*(x_2)$ is precisely a path in $b_1 = b_2$ and to get a path $x_1 = x_2$ we only need a path $a_1 = a_2$. Note that we have a path in $f(a_1) = f(a_2)$ from paths in this horseshoe of equalities
\begin{center}
\begin{tikzcd}
    f(a_1) \equiv f \circ g^* (x_1) \arrow[equal, r] & g \circ f^* (x_1) \equiv g(b_1) \arrow[d,equal] \\
    f(a_2) \equiv f \circ g^*(x_2) \arrow[r,equal] & g \circ f^*(x_2) \equiv g(b_2) 
\end{tikzcd}
\end{center}
where the horizontal paths come from the $\omega_i$ and the vertical equality comes from the path in $b_1 = b_2$. Since $f$ is an embedding, this path in $f(a_1) = f(a_2)$ gives us a path in $a_1 = a_2$ and we have the first part of the embedding.

Now we need to show is-equiv($\ap{f^*})$. Note that $$\ap{f^*} : (a_1,b_1,\omega_1) = (a_2,b_2,\omega_2) \to b_1 = b_2 $$ is a projection map, so we have to find an inverse. We start with a path $p_b : b_1 = b_2$. We first must find a path $p_a: a_1 = a_2$. Since $\ap{f}$ is an equivalence, finding a path in $a_1 = a_2$ is equivalent to finding a path in $f(a_1) = f(a_2)$. We construct one via:
\begin{center}
\begin{tikzcd}
    f(a_1) \arrow[equal,r,"\omega_1"] & g(b_1) \arrow[d,equal,"\ap{g}(p_b)"] \\
    f(a_2) \arrow[r,equal,"\omega_2",swap] & g(b_2)
\end{tikzcd}
\end{center}
We have the horizontal maps by assumption and the vertical map given by $\ap{g}(p_b)$. This gives a path $p_a : f(a_1) = f(a_2)$. The commuting square given by inserting $p_a$ in the above diagram gives us a proof $\omega_{(a,b)}$.

It is clear that $\ap{f^*} \circ \ap{f^*}\inv = \id_{b_1 = b_2}$. To complete the proof we only need show that $\ap{f^*}\inv \circ \ap{f^*} = \id_{(a_1,b_1,\omega_1) = (a_2,b_2,\omega_2)}$. Let us consider the image of $(p,q,\omega)$ under $\ap{f^*}\inv \circ \ap{f^*}$, which we call $(p\pr,q\pr,\omega\pr)$.

We need to show $p = p\pr, q = q\pr,$ and $\omega = \omega\pr.$ The second of these is definitional. Then showing $p = p\pr$ is equivalent to showing $\ap{f}(p) = \ap{f}(p\pr)$. By definition, the type $\ap{f}(p\pr) = \omega_1 \cdot \ap{g}(p_b) \cdot \omega_2$ is inhabited. Similarly, $\omega$ gives an inhabitant of $\ap{f}(p) = \omega_1 \cdot \ap{g}(q) \cdot \omega_2$, thus the type $\ap{f}(p) = \ap{f}(p\pr)$ is inhabited, as desired. By the same arguments, the commuting squares giving $\omega$ and $\omega\pr$ are equal, and we have our result.
\end{proof}

\begin{lemma}\label{lemma: subgroup of free abelian}
A subgroup of a free abelian group is free.
\end{lemma}

\begin{lemma}\label{lemma: group extension}
If there is a short exact sequence of abelian groups \TODO{Abelian might be necessary for the SESs.} of the form $$ \1 \to A \to B \to C \to \1$$then $(b_0 =_B b_0) = \Sigma_{c : (c_0 =_C c_0)} (a_0 =_A a_0)$.
\end{lemma}

\section{Free Groups}
Note that none of the proof above used the fact that anything was abelian,\TODO{It just might not be interesting if the groups aren't abelian.} except in the existence of a surjection. This should generalize to other things, then, such as free groups, or anything defined freely by use of a surjection. This are of interest in algebraic structures.

\section{Next}
\begin{itemize}
    \item Prove the lemmas
    \item Give a more general categorical exposition of the notion of a Serre class, which "generalizes 0"
    \item Look at the other properties, such as Tor.
\end{itemize}


\begin{comment}
\section{Definition of $\z_k$ with $\sone$}
We define a finite cyclig group of order $k : \n$, written $\z_k$, giving a map $\n \to \Univ$. Note that $B(\z_k)$ (can be?) defined as:
\begin{itemize}
\item $\base : \z_k$
\item $\Loop : \base =_{\z_k} \base$
\item $\cone : \Loop^k = \Loop$
\end{itemize}
We define $\z_k \defequiv || B(\z_k) ||_1$


It is straightforward (though probably difficult formally?) to show that $$ \1 \to \sone \to \sone \to \z_k \to \1 $$ is exact.


\section{Finitely Generated Abelian Groups}

We want to get at the concept of finitely generated abelian groups. Classically, $G$ is finitely generated precisely when there is a short exact sequence of the form $$ \0 \to \z^m \to \z^n \to G \to \0.$$
We can define an object that will, ideally, behave similarly. We can define a map $\otimes_\_ \sone: \n \to \Univ$ which maps $0 \mapsto \1$ and $n+1 \mapsto \sone \times \otimes_n \sone$. We'd like to define a subuniverse $\Univfg$ which behaves like finitely generated abelian groups, with the eventual goal of showing that $\pi_k (\mathbb{S}^n)$ is in this subuniverse for all $n,k : \n$.

The short exact sequence that behaves similarly to the classical version will then have the following form:
$$\1 \to \otimes_m \sone \to \otimes_n \sone \to B \to \1$$
So we say $B$ is a ``finitely generated abelian group'' if there is some short of exact sequence of this form. If $\omega$ is such a witness, then $(B,\omega) : \Univfg$.

We define $\Univfg$, then to be the type $$ \Univfg \defequiv \Sigma_{B : \Univ} || \SES(m,n,B) ||_{-1}$$
where $\SES$ is defined to be the type 
$$\SES \defequiv \Sigma_{m,n : \n} \Sigma_{B : \Univ} \Sigma_{f : \otimes_m \sone \to \otimes_n \sone} \Sigma_{g: \otimes_n \sone \to B} \textrm{is-exact}(m,n,B,f,g)$$
and $\isexact$ is a map \textcolor{red}{(needs to be defined)} that shows a sequence to be short exact.

\section{Questions}

Main question:
\begin{itemize}
\item Define isSerreClass (should be a proposition)
\item Is $\Univfg$ a Serre class? i.e. 
\item Can we prove the universal property that if $\1 \to \otimes_m \sone \to \otimes_n \sone \to B \to \1$ is exact, then for any $A : \Univ$ such that $\1 \to \otimes_m \sone \to \otimes_n \sone \to A$ is exact, there is a unique embedding of $B \inj A$?
\end{itemize}

\end{comment}



\end{document}
